\documentclass{letter}

\address{Department of Physics and Astronomy\\Stony Brook University\\Stony Brook NY 11794\\United States}

\signature{Will Farr}

\begin{document}

\begin{letter}{Dr. Leslie Sage\\Springer Nature\\One New York Plaza, Suite 4600\\New York NY 10004-1562\\USA}

\opening{Dear Dr. Sage:}

Please find enclosed a submission for consideration as a Nature Letter, titled
``A Future Percent-Level Measurement of the Hubble Expansion at Redshift 0.8
With Advanced LIGO.''  As you will see from the manuscript, the goal of this
work is to point out how a recently-discovered feature in the mass distribution
of merging binary black holes enables precision cosmology at redshift $z \simeq
0.8$.

General relativity predicts the amplitude of a gravitational wave signal; such
signals are ``standard sirens'' that enable a direct measurement of the
luminosity distance to the source.  In contrast to electromagnetic astronomy,
the challenge with gravitational wave sources is to measure the redshift to the
source.  There are a number of ways to do this (see references in the
manuscript, including the ``counterpart method,'' published in Nature by the
LIGO and Virgo collaborations following GW170817).

Here we propose to exploit an ``absorption'' feature in the mass distribution of
merging binary black holes to infer redshifts without any corresponding
electromagnetic measurements.  Hints of a maximum mass of merging black holes
are apparent in the first gravitational wave transient catalog; a physical
process, the ``pair instability supernova,'' has been proposed to reduce the
remnant mass of or disrupt a high mass star completely, leaving no black holes
in binaries above a mass $\simeq 45 \, M_\odot$.  Because gravitational wave
detectors measure a redshifted mass, $m_\mathrm{detector} = m_\mathrm{source}
(1+z)$, tracking the detector-frame mass at which the black hole mass
distribution tails off measures the redshift of a population of mergers.

We are not the first to propose using features in a mass distribution to measure
redshifts in a gravitational wave detector (the idea goes back to Chernoff \&
Finn (1993)); but until now all such proposals exploited the narrow range of
merging \emph{binary neutron star} masses.  Such mergers are not suitable for
cosmography in the current era because the reach of present detectors to neutron
star mergers, $\mathcal{O}\left( 100 \, \mathrm{Mpc} \right)$, is not sufficient
for the mass to redshift meaningfully.  In contrast, Advanced LIGO and Virgo
operating at design sensitivity can detect a merger of two black holes near the
pair instability limit at redshifts $z \simeq 1.5$, so precision cosmography is
possible with current detectors.

We predict that with five years of Advanced LIGO and Virgo observations the
Hubble expansion can be constrained to better than 3\% at $z \simeq 0.8$.
Besides the intrinsic excitement in such an independent measurement of the
expansion rate of the universe (see manuscript), we believe that this method
will become the dominant cosmological tool in the emerging field of
gravitational wave cosmology.  The measurement relies on understanding and
calibrating the amount of intrinsic evolution in the pair instability scale
(which is degenerate with the redshift measurement); by analogy to type Ia
supernovae, remnants near the pair instability gap become standard\emph{izable}
sirens.  Thus, we expect this result to drive a large portion of the near-term
research agenda of \emph{two} important astronomical fields: gravitational wave
cosmology \emph{and} the evolution of massive stars in binaries.

We note that the submitted manuscript is not yet compliant with the Nature
Letter formatting standards (particularly with respect to the number of
references).  We do not believe that it would be too difficult to bring it in
line with your requirements, however, should you be interested in publishing it.

We thank you for your careful consideration of this manuscript.

\closing{Sincerely,}

\end{letter}

\end{document}
